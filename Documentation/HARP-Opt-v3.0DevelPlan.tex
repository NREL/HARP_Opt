\documentclass[12pt]{article}

\usepackage{setspace}
\usepackage{fullpage}
\usepackage{graphicx}
\usepackage[section] {placeins}
\usepackage{hyperref}
\usepackage{color}

\begin{document}

\title{HARP-Opt v3.0 Development Plan}
\author{Danny Sale and Michael Lawson}

\maketitle

%\begin{abstract}
%The abstract text goes here.
%\end{abstract}

\section{Objective}
This document describes the development for HARP-Opt version 3.0, which will be released by September 30, 2012. The objectives of this development effort are:
\begin{list}{$\bullet$}{\leftmargin=3em \itemindent=0em \rightmargin=3em}
	\item Make the WT-Perf FORTRAN program used by HARP-Opt "callable" as a MATLAB function (note that WT-Perf will still be implemented in FORTRAN). This will remove the need for computationally expensive disk input-output each time WT-Perf is run by HARP-Opt.
    \item Remove the graphical users interface (GUI) from the HARP-Opt code. Make the code operate using a text input file.
    \item Develop a stand alone GUI, referred to as HARP-GUI herein, that writes the text input file for HARP-Opt
    \item Implement a patter search optimization algorithm to improve repeatability of the results
\end{list}
These development objectives are described in more detail in the following section.

\section{Discussion of Development Objectives}

\subsection{Interoperate the callable version of WT-Perf into HARP-Opt}
The current version of HARP-Opt runs WT-Perf as an external program. This requires HARP-Opt to write WT-Perf input files and read WT-Perf output files from disk. This is a computationally expensive operation that contributes significantly to the total HARP-Opt run time.
 
Andrew Platt is currently developing a version of WT-Perf that is callable as a FORTRAN module/function (details are to be determined). \href{http://www.mathworks.com/help/techdoc/matlab_external/f29502.html}{\color{blue}The MathWorks} provides tools to make this possible.

A significant amount of the HARP-Opt code is dedicated to file input-output, so the ability to call WT-Perf from MATLAB should significantly simplify the code.

\subsection{Platform Independence}
The new release of HARP-Opt should be platform independent (i.e. able to run on any operating system). This will require several things:
\begin{list}{$\bullet$}{\leftmargin=3em \itemindent=0em \rightmargin=3em}
    \item Sub-programs that are called by HARP-Opt will need to be compilable from any operating system.
    \item Anything else?
\end{list}
%\end{list}

\subsection{No GUI for the Main HARP-Opt Code!}
GUIs and graphics of any kind should be completely removed from the code. I have already removed the status bar from the airfoil interpolation operation. The graphical output that is displayed during the optimization run should be replaced with a text based output. This should simplify the code. This will also allow the program to be run from the command line and for batch jobs consisting of many HARP-Opt runs to be on high-performance-computing (HPC) resources.

\subsection{GUI to Write Input Files}
Develop a HARP-Opt pre-processor GUI (HARP-Opt-Pre, need a better name, but this will work for now) to write an input file for the HARP-Opt code. HARP-Opt-Pre may be developed so it calls the HARP-Opt code, however, the code of the HARP-Opt-Pre will be completely separate from the HARP-Opt code. 

\section{Long Term Development Goals}
\begin{list}{$\bullet$}{\leftmargin=3em \itemindent=0em \rightmargin=3em}
    \item \textbf{Open-Source:} Move to all open-source packages. Coding will be in Python and optimization will by in some open-source optimization package, such as \href{http://www.pyopt.org/}{\color{blue}pyOpt} or  \href{http://dakota.sandia.gov/software.html}{\color{blue}DAKOTA}
    \item \textbf{Fatigue Analysis Capabilities:} Consider fatigue loads in the optimization process. Maybe
    the easiest way to do this is to consider only the edgewise gravity loads...which we know the frequency
    from the rotor speed.  Combined with the flow PDF, we should be able to estimate fatigue loads.
	\item Andrew Ning has some improvements to WT-Perf that will allow for the use of gradient search methods to be used.
    \item Migrate HARP-Opt to \href{https://github.com/NREL/HARP_Opt}{\color{blue}GitHub} and suggest using \href{http://nvie.com/posts/a-successful-git-branching-model/}{\color{blue}Vincent Driessen's git branching model} for further development and collaboration
    
    




\end{list}

\end{document}